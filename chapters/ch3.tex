% Chapter 3 Notes
\section{Grand Canonical Ensemble}
The grand canonical ensemble is a collection of systems with the same volume,
temperature, and chemical potential. The ensemble can be thought of as a
collection of systems which have been equilibriated w.r.t chemical potential and
temperature by placing the ensemble in a heat bath and particle bath. The
ensemble once equilibriated is then isolated. Each system within the ensemble is
free to exchange particles or energy though.

With the concept of the GC ensemble established, these equations are easily
acertained.
\begin{align*}
	\sum_N \sum_j{a_{Nj}} & = A \\\\
	\sum_N \sum_j{a_{Nj}E_{Nj}} & = E_{tot} \\\\
	\sum_N \sum_j{a_{Nj}N} & = N_{tot}
\end{align*}
Note that $a_{Nj}$ is the occupancy number of systems with $N$ particles and the
$j$-th energy state. Like the canonical ensemble the GC ensemble is a system in
the microcanonical ensembe so we can use the principle of equal \textit{a
priori} probabilities to the distributions of occupancy numbers. The number of
states given a distribution $\bar{a_{Nj}}$ is
\begin{equation*}
	W(\bar{a_{Nj}}) = \frac{A!}{\prod_N{}\prod_j{a_{Nj}!}}.
\end{equation*}
The solution to maximizing $W$ with the constrants given is nearly identical to
the C ensemble method and simply becomes
\begin{align*}
	a_{Nj}^{*}&= e^{-\alpha}e^{-\beta E_{Nj}}e^{-\gamma N} \text{ thus,} \\\\
	P_{Nj}&= \frac{a_{Nj}^*}{A} = \frac{e^{-\alpha}e^{-\beta E_{Nj}}e^{-\gamma
	N}}{\sum_N \sum_j{e^{-\alpha}e^{-\beta E_{Nj}}e^{-\gamma N}}}
\end{align*}
We note that the denominator is a sum of all states with a two biasing factors
based on $\beta E_{Nj}$ and $\gamma N$. Like the C ensemble this will be called
the partition function of the GC ensemble. We label it $\Xi$.

\subsection{Finding $\beta$ and $\gamma$}
Lets think of the GC ensemble as collections of canonical ensembles with varying
numbers. Each ensemble is in thermal equilibrium since the entire GC is. If we
suddenly prevent molecular transport. Then $\Xi$ breaks down to
\begin{equation*}
	\Xi = e^{\gamma N} \sum_j{ e^{-\beta E_j}} \text{, } \forall N.
\end{equation*}
This is the canonical partition function with an extra constant which can be
ignored since this constant in probabilities would be applied to both the
numerator and denominator. Thus, $\beta$ must be the same as the canonical
ensemble.

$\gamma$ is a little more involved to derive. If we take the full derivative of
$\ln \Xi$, we find
\begin{equation*}
	df = \left(\frac{\partial f}{\partial \beta}\right)_{\gamma,E_{Nj}} d \beta
	+ \left(\frac{\partial f}{\partial \gamma}\right)_{\beta,E_{Nj}} d \gamma
	+ \left(\frac{\partial f}{\partial \gamma}\right)_{\beta,\gamma,E_{N_ik\neq
	Nj}} d E_{Nj}.
\end{equation*}
This ends up being
\begin{align*}
	df &= -\bar{E} d \beta - \bar{N} \gamma + \beta \bar{p} dV \\
	d(f + \beta \bar{E} + \gamma \bar{N}) &= \beta d \bar{E} + \beta \bar{p} dV
	+ \gamma d\bar{N}.
\end{align*}
For multicomponent systems this is the enemble equivalent to the differential
form of entropy. Then $\gamma$ must be
\begin{equation*}
	\gamma = -\frac{\mu}{kT}.
\end{equation*}
An important note is that $\gamma$ is essential another energy scale (chemical
potential) compared to thermal energy.

\subsection{Other Points}
$\Xi$ can be written as,
\begin{equation*}
	\sum_N{Q(N)e^{\beta\mu N}}
\end{equation*}

Often times we see $e^{\beta\mu}$ as $\lambda$ which is called the
fugacity and is the absolute activity of a state. This can be seen since
\begin{align*}
	\mu &= kT \ln \lambda \\
	\Delta \mu &= kT \ln(\lambda_2 / \lambda_1)
\end{align*}

The function $pV$ is the natural thermodynamic function for the GC ensemble. The
proof is simple,
\begin{align*}
	S &= \frac{\bar{E}}{T} - \frac{\bar{N}\mu}{T} + k\ln\Xi \\
	G &= \mu N = E + pV - TS \\
	pV &= \mu N - E + TS = \mu N - E + T ( \frac{\bar{E}}{T} -
	\frac{\bar{N}\mu}{T} + k\ln\Xi)
\end{align*}

One final note, the summation over $N$ goes to infinity since in the limit of
infinite systems there is a nonzero chance of at least one system having
infinite particles. However, we will see that practically speaking the summation
could be cut off for reasons of fluctuations.

\section{Gibb's Ensemble}\label{sec:Gibb's Ensemble}
Gibb's ensemble is a collection of systems that have been thermally
equilibriated and have impermeable but flexible walls. Thus pressure is also
equal each system. Number is also fixed by impermeability. $\Delta$ is the
standard denotation of the partition function which is,
\begin{equation*}
\sum_E \sum_V \Omega(V, E)e^{ -\beta E}e^{\beta pV}
\end{equation*}
True to its name the natural thermodynamic function for the Gibb's ensemble is
Gibb's energy.

\section{MicroCanonical Ensembles}%
\label{sec:MicroCanonical}

\subsection{Universality of Microcanonical Ensemble}
For each ensemble, when summing over some combination
of energy levels, volumes, or number, we find the degeneracy by the
microcanonical partition function $\Omega$. Thus all other ensembles can be
thought of as deviations from the microcanonical. The exponential terms are
biasings based on intensive properties and the sums represent summations over
the extensive properties.

\subsection{Boltzman Equation}
If we start from the definition of entropy derived from the GC partition
function, we find the famous Boltzman equation which is actually on Boltzman's
tomb, and gives intuition as to the nature of entropy.
\begin{align*}
	S &= k\ln\Xi + k \left( \sum_{N,j}{\beta E_{Nj} \frac{e^{-\beta E_{Nj}}
		e^{-\gamma N}}{\Xi}} +
		\sum_{N,j}{\gamma \frac{Ne^{-\beta E_{Nj}} e^{-\gamma N}}{\Xi}}
	\right) \\
	S &= k\ln\Xi + k \sum_{N,j}{\left( \beta E_{Nj} + \gamma N \right)
		\frac{e^{-\beta E_{Nj}} e^{-\gamma N}}{\Xi}}
\end{align*}
Recall that,
\begin{align*}
	\frac{a_{Nj}^*}{A} &= \frac{e^{-\beta E_{Nj}} e^{-\gamma N}}{\Xi}\\
	\text{and}\\
	\ln \frac{a_{Nj}^*}{A} &= \ln a_{Nj}^* - \ln A =
		\ln \left(e^{-\beta E_{Nj}} e^{-\gamma N} \right) - \ln\Xi\\
	&= -(\beta E_{Nj} + \gamma N + \ln\Xi)
\end{align*}
Thus,
\begin{align*}
	S &= k\ln\Xi - k \sum_{Nj}{(\ln{a_{Nj}^*} - \ln{\Xi} + \ln{A})
	\frac{a_{Nj}^*}{A}}\\
	  &= k\ln{A} - \frac{k}{A} \sum_{N,j}{a_{Nj}^*\ln{a_{Nj}^*}}
\end{align*}
This is for each system in the GC ensemble. The entire entropy is then,
\begin{align*}
	S &= kA\ln{A} - k\sum_{N,j}{a_{Nj}^*\ln{a_{Nj}^*}}\\
	  &= k \ln{\frac{A}{\sum_{N,j}{a_{Nj}^*\ln{a_{Nj}^*}}}}\\
	  &= k \ln{W(\vec{a_{Nj}^*})}\\
	  &= k \ln{\Omega}.
\end{align*}

\section{Fluctuations}\label{sec:Fluctuations}
For the assumption that large systems only assume states in their ensemble
average, fluctuations around the mean must be small. If they were not, then
statistical mechanics would predict that we should observe macroscopic
fluctuations in state. Two canonical ways of showing that fluctuations are
indeed small (as compared to the mean) the energy of the canonical and number of
the grand canonical ensemble will be shown. The most important equation when
dealing with fluctuations or second moments around the mean is,
\begin{equation*} 
	\sigma_M^2 = \overline{\left( M - \bar{M}\right)^2} = \overline{M^2} - \bar{M}^2
\end{equation*}

\subsection{Energy Fluctuations}
For the canonical ensemble $\overline{E^2}$ is
\begin{align*}
	\sum_j{E_j^2 P_j} &= \frac{1}{Q}\sum_j{E_j^2 e^{-\beta E_j}}\\
					 &= \frac{-1}{Q} \frac{\partial}{\partial \beta}
					 \sum_j{E_j e^{-\beta E_j}}
\end{align*}
The last sum is just the average energy without the partition function's
normalization, so
\begin{equation*}
	\overline{E^2} = \frac{-1}{Q} \frac{\partial}{\partial \beta} (\bar{E}Q).
\end{equation*}
Via product rule we get
\begin{align*}
	\overline{E^2} &= -\frac{\partial\bar{E}}{\partial \beta} - \bar{E}
	\frac{\partial\ln{Q}}{\partial \beta}\\
			  &= kT^2 \frac{\partial\bar{E}}{\partial\beta} + \bar{E}^2.
\end{align*}
Finally, via the definition of the varience,
\begin{align*}
	\sigma_E^2 &= kT^2 \left( \frac{\partial\bar{E}}{\partial T} \right)_{N,V}
			   &= kT^2 C_v
\end{align*}
Compared to the average energy then the average fluctuations are,
\begin{equation*}
	\frac{\sigma_E}{\bar{E}} = \frac{(kT^2 C_v)^{\frac{1}{2}}}{\bar{E}}
\end{equation*}
For an ideal gas $\bar{E}$ scales as $O(NkT)$ while $C_v$ scales as $O(Nk)$.
Thus the fluctuations are proportional to $O(N^{\frac{1}{2}})$ and in a large
system are very small. It is worth noting the same result can be achieved from a
Taylor expansion of the probability distribution around $\bar{E}$.

\subsection{Density Fluctuations}
For the GC ensemble we shall look at number fluctuations. The fluctuations are
\begin{align*}
	\sum_{N,j}{N^2 P_{Nj}} &= \frac{1}{\Xi}\sum_{N,j}{N^2 e^{-\beta E_{Nj}}
	e^{-\gamma N}} = \frac{-1}{\Xi} \frac{\partial}{\partial\gamma}
	\sum_{Nj}{N e^{-\beta E_{Nj}}e^{-\gamma N}}\\
						   &= \frac{-1}{\Xi} \frac{\partial}{\partial\gamma}
						   (\bar{N}\Xi)\\
						   &= \frac{\partial\bar{N}}{\partial\gamma} -
						   \bar{N} \frac{\partial\ln{\Xi}}{\partial\gamma}\\
						   &= kT \left( \frac{\partial\bar{N}}{\partial\mu}
						   \right)_{V, T} + \bar{N}^2
\end{align*}
This looks very similar to the canonical example and should because it is
analogous. All we need now is the thermodynamic relationship,
\begin{equation*}
	\left( \frac{\partial\mu}{\partial N} \right)_{V,T} =
	- \frac{V^2}{N^2} \left( \frac{\partial p}{\partial\mu}\right)_{N,T}.
\end{equation*}
The fluctuations are,
\begin{align*}
	\sigma_N &= \left( \frac{\bar{N}^2 kT\kappa}{V}\right)^{\frac{1}{2}}\\
	\frac{\sigma_N}{\bar{N}} &= \left( \frac{kT\kappa}{V} \right)^{\frac{1}{2}}.
\end{align*}
Here $\kappa$ is the isothermal compressibility which is $\frac{1}{p}$ for the
ideal gas. Then the fluctuations for the ideal gas is once again the property to
the $\frac{-1}{2}$. Furthermore, like before the Taylor expansion can be
performed to get similar results.

For number fluctuation, this has measurable effects on the Raleigh scattering.
The functions make it so that the scattering is a function of wavelength to the
$-4$th. Therefore, in our atmosphere, blue light gets scattered much more and
the sky is blue and sunsets are red.
