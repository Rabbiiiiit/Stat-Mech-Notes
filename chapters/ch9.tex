Like other properties such as pressure, temperature, and free energy can be
calculated from the partition function of a system, chemical equilibrium can be
derived from them as well.

\textit{Note: }that examples present in the book will not be reproduced because
they are straight forward and just involve using the equations from
Section~\ref{sec:chemeqetpf} and those of the partition functions of different
molecule types as previously derived.

\section{Chemical Equilibrium in Terms of Partition Functions}%
\label{sec:chemeqetpf}
We first note that chemical equilibrium in a reaction or phase change occurs
when,
\begin{equation*}
	\mu_I = \mu_{II},
\end{equation*}
or when chemical potentials are equal between the products and reactants or
between the two phases. This holds for multiple reactions and multiphase
behavior as well. One can also derive the necessary relations from a free energy
argument. Take an arbitrary homogeneous gas reaction,
\begin{equation*}
	\nu_A A + \nu_B B \leftrightarrow \nu_C C + \nu_D D.
\end{equation*}
We assume that the reaction is at equilibrium in a rigid thermostatted
container. By definition of the canonical ensemble this means that the Helmholtz
free energy is at a minimum. By the conservation of mass and the definition of
equilibrium we know
\begin{equation*}
	\nu_C C + \nu_D D - \nu_A A - \nu_B B = 0.
\end{equation*}
Taking the definition of the Helmholtz free energy as,
\begin{equation*}
	\d{A} = -S\d{T} - p\d{V} + \sum_{j}{\mu_j \d{N_{j}}},
\end{equation*}
and using the fact that we are at constant $T$ and $V$, we get
\begin{equation*}
	\d{A} = \sum_{j}{\mu_j \d{N_{j}}} = {\left(\sum_{j}{\nu_j \sigma_j
	\mu_{j}}\right)}\d{\lambda},
\end{equation*}
where $\sigma_j \nu_j \d{\lambda} = \d{N_{j}}$ with $\sigma_j$ being $1$
for products and $-1$ for reactants. Since at equilibrium we are at a
minimum with respect to $A$, we know that $\partial A/ \partial \lambda$ must be
zero, and
\begin{equation*}
	\sum_{j}{\sigma_j \nu_j \mu_j} = 0.
\end{equation*}
Now to put statistical mechanics back into it we need an expression for the
partition function for the multicomponent system.  If the systems is an ideal
gas, the partition function of the system is just the multiplication of the
\begin{align*}
	\Q_{tot} &= \Q_{N_{A}} \Q_{N_B} \Q_{N_C} \Q_{N_{D}}\\
			&= \frac{q_{A}^{N_A}}{N_A !} \frac{q_{B}^{N_B}}{N_B !}
			   \frac{q_{C}^{N_C}}{N_C !} \frac{q_{D}^{N_D}}{N_D !}.
\end{align*}
Using the previously derived relation,
\begin{align*}
	\mu_A &= -kT {\left(\frac{\partial \ln{\Q}}{\partial N_{A}}\right)}_{N_j, V,
	T}
			= -kT {\left(\partial \sum_{i}^{A, B, C, D}
					{\ln{\left[\frac{q_{i}}{N_i !}}\right]} / \partial N_A
			\right)}
			 = -kT {\left(\partial \ln{\left(\frac{q_{A}}{N_A !}\right)} /
			 \partial N_A \right)}\\
		  &= -kT \partial(N_{A} \ln{q_{A}} - N_{A}\ln{N_{A}} + N_{A}) / \partial N_A
		  = -kT {\left( N_A \ln{\left[\frac{q_A}{N_{A}}\right]} + N_A \right)} /
		  \partial N_A \\
		  &= -kT {\left( \ln{\left[ \frac{q_A}{N_A} \right]} - 1 + 1 \right)}\\
		  &= -kT \ln{\left[ \frac{q_A}{N_A} \right]}.
\end{align*}
The second to last line uses the product rule. This results states that the
chemical potential of species A is only dependent on species A which is what one
expects for an ideal gas. We know that
\begin{align*}
	\sigma_j \nu_j \mu_j &= -kT \sigma_j \nu_j \ln{\left[ \frac{q_j}{N_j}
	\right]} = \ln{\left[ \frac{q_j}{N_j} \right]}^{\sigma_j \nu_{j}}\\  
	\sum_{j}{\sigma_j \nu_j \mu_j} &= \nu_D \mu_D + \nu_C \mu_C - \nu_A \mu_A -
	\nu_B \mu_B =0 \\
		&= \sum_{j}{\sigma_j \nu_j \ln{\left[ \frac{q_j}{N_j} \right]}}
		= \sum_{j}{\ln{\left[ \frac{q_j}{N_j} \right]}^{\sigma_j \nu_{j}}}\\
		&= \ln{\left[ {\left(\frac{q_D}{N_{D}}\right)}^{\nu_D}
		{\left(\frac{q_C}{N_{C}}\right)}^{\nu_C}
		{\left(\frac{N_B}{q_{B}}\right)}^{\nu_B}
		{\left(\frac{N_A}{q_{A}}\right)}^{\nu_A} \right]} = 0\\
		&= \ln{\left[
			\frac{q_{D}^{\nu_D}q_{C}^{\nu_C}}{q_{B}^{\nu_B}q_{A}^{\nu_A}}
			\right]} +
			\ln{\left[
			\frac{N_{B}^{\nu_B}N_{A}^{\nu_A}}{N_{D}^{\nu_D}N_{C}^{\nu_C}}
			\right]} = 
			\ln{\left[
			\frac{q_{D}^{\nu_D}q_{C}^{\nu_C}}{q_{B}^{\nu_B}q_{A}^{\nu_A}}
			\right]} -
			\ln{\left[
			\frac{N_{D}^{\nu_D}N_{C}^{\nu_C}}{N_{B}^{\nu_B}N_{A}^{\nu_A}}
			\right]} = 0 \\
	\frac{q_{D}^{\nu_D}q_{C}^{\nu_C}}{q_{B}^{\nu_B}q_{A}^{\nu_A}} &= 
		\frac{N_{D}^{\nu_D}N_{C}^{\nu_C}}{N_{B}^{\nu_B}N_{A}^{\nu_A}} 
\end{align*}
For an ideal gas, the partition function is of the form $f(T) V$. Thus, a purely
temperature dependent equilibrium constant $K_c (T)$ can be defined as
\begin{equation*}
	K_c (T) = \frac{\rho_{D}^{\nu_D}\rho_{C}^{\nu_C}}
	{\rho_{B}^{\nu_B}\rho_{A}^{\nu_A}} =
	\frac{{(q_{D}/V)}^{\nu_D}{(q_{C}/V)}^{\nu_C}}
	{{(q_{B}/V)}^{\nu_B}{(q_{A}/V)}^{\nu_A}}.
\end{equation*}
The ideal gas law $p_j = \rho_j kT$can be used to derive an expression for a pressure based
equilibrium constant $K_p (T)$,
\begin{equation*}
	K_p(T) = \frac{p_{D}^{\nu_D}p_{C}^{\nu_C}}{p_{B}^{\nu_B}p_{A}^{\nu_A}} =
	(kT)^{\sum_{j}{\sigma_j \nu_{j}}} K_c (T).
\end{equation*}

\section{Thermodynamic Tables}%
\label{sec:chemeqtt}
Tables of experimentally determined thermodynamic properties exist when can give
us the desired equilibrium constants. Taking the ideal gas relation,
\begin{equation*}
	\mu(T,p) = \mu_0 (T) + kT\ln{p}.
\end{equation*}
Using the fact that $\sum_{j}{\sigma_j \nu_j \mu_j} = 0$, we have
\begin{align*}
	\sum_{j}{\sigma_j \nu_j kT\ln{p_{j}}} + \sum_{j}{\sigma_j \nu_j
	\mu_{0,j}(T)} &= 0\\
	kT\ln{K_{p}} = - {\Delta \mu_{0}}.
\end{align*}

Looking once more at the expression developed for $\mu$, in terms of $q$, we can
put $\mu$ into terms of $(T, p)$.
\begin{align*}
	\mu &= -kT\ln{\left[\frac{q}{N}\right]} = 
	-kT\ln{\left[\frac{q}{V} \frac{V}{N}\right]}\\
		&= -kT\ln{\left[\frac{q}{V} \right]} + kT\ln{[\rho]} =
		-kT\ln{\left[\frac{q}{V} \right]} + kT\ln{\left[ \frac{p}{kT} \right]} \\
		&= -kT\ln{\left[\frac{q}{V} kT \right]} + kT\ln{[p]} \\
	\mu_0 (T) &= -kT\ln{\left[\frac{q}{V} kT \right]}.
\end{align*}
Here one should note that the logarithm in the last line has units of pressure.
\subsection{Zero of Energy}
One important consideration when using thermodynamic tables is the zero of
energy. Previously we have set the translational and rotational ground states to
0 energy, the vibrational zero point energy to be the bottom of the potential
well, and the electronic zero to be the energy of complete atomic separation.
Experimentally, the complete ground state is taken to be zero energy. This
requires changing the electronic and vibrational partition functions from what
we have previously derived.

We first take the electronic partition function,
\begin{equation*}
	q_{elec} = \omega_{e1} e^{-\varepsilon_{e1}/kT} + \omega_{e2}
	e^{-\varepsilon_{e2}/kT} + \cdots
\end{equation*}
Using the same convention that the zero energy is the electrically neutral
separated atoms then,
\begin{align*}
	\varepsilon_{e_2} &= -D_e \\
	\varepsilon_{e_2} &= -D_e + \Delta\varepsilon_{12} \\
	q_{elec} &= \omega_{e1} e^{D_e/kT} + \omega_{e2}
	e^{D_e/kT}e^{-\Delta\varepsilon_{12}/kT} + \cdots \\
			 &= e^{D_{e}/kT} (\omega_{e_{1}} + \omega_{e_2}
			 e^{-\Delta\varepsilon_{12}/kT} + \cdots) \\
			 &= e^{D_e/kT} q_{elec}^0.
\end{align*}
By factoring out the $e^{D_{o}/kT}$, the partition function becomes a constant
times the partition function with zero of energy the ground state as
$\omega_{e_{1}}$ is multiplied by $1$ or $e^0$. We can similarly take the
vibrational partition function,
\begin{equation*}
	q_{vib} = \prod_{j}{\frac{e^{-\Theta_{\nu_{j}}/2T}}
	{(1 - e^{-\Theta_{\nu_{j}}/2T})}}, where \Theta_{\nu_j} = \frac{h\nu}{k}
\end{equation*}
and remove the part from the non-zero ground state. Here it is important to note
from the derivation in Section~\ref{sec:diatomicvib} that the numerator comes
from the $1 /2$ term in the energy levels which is a result of having a nonzero
ground state. Thus,
\begin{equation*}
	q_{vib} = e^{-\Theta_{\nu_{j}}/2T}\prod_{j}{(1 - e^{-\Theta_{\nu_{j}}/2T})}
	= e^{-\Theta_{\nu_{j}}/2T} q_{vib}^0.
\end{equation*}

From these two equations we can redefine the molecular partition function as,
\begin{align*}
	q &= q_{trans}^0 q_{rot}^0 q_{vib}^0 q_{elec}^0 e^{(D_e - 1/2
		\sum_{j}{h\nu_{j}})/kT} = q_{trans}^0 q_{rot}^0 q_{vib}^0 q_{elec}^0
		e^{D_o /kT}.
\end{align*}
The substitution to $D_o$ is graphically explained in
Figure~\ref{fig:diatomicelectronic}. From the graphical argument and the
definition of $D_o$, $-D_{o}$ can been seen to be the total ground state energy
from our original convention. Thus, for any zero energy convention the partition
function is
\begin{equation*}
	q = q^0 e^{-\varepsilon_{0}^0 /kT}.
\end{equation*}
Substituting this into the equation for chemical potential we have,
\begin{align*}
	\mu &= -kT\ln{\left[\frac{q}{V} kT \right]} + kT\ln{[p]} \\
		&= -kT\ln{\left[e^{-\varepsilon^0_0}\frac{q^0}{V} kT \right]} +
		kT\ln{[p]} \\	
	\mu - \varepsilon^0_0 &= -kT\ln{\left[\frac{q^0}{V} kT \right]} +
	kT\ln{[p]}.
\end{align*}
An important fact here is that $\mu \to \varepsilon$ as $T \to 0$. If we treat
$\mu - \varepsilon^0_0$ as the relative chemical potential, then it goes to 0 at
$T \to 0$. Multiplying by Avogadro's number, we obtain,
\begin{equation*}
	G^0 - E^0_0 = -RT \ln{\left[\frac{q^0}{V} kT \right]} + kT\ln{[p]}.
\end{equation*}
This means that $E^0$ is the standard free energy at 0 K which means $G^0 -
G^0_0$ is an equally appropriate choice. Likewise, since $H = E + pV$ and $p \to
0$ as $T \to 0$, then $H^0_0 = E^0_0 $. Thus, these three notations are
equivalent and possible.  An important comment on convention is required to use
the derived quantities above. An elements energy is considered 0 at 0 K if the
element is in the same physical state that is stable at $25^{\circ}$C and 1 atm.
