\section{Formulation of the Classical Partition Function}%
\label{sec:cpf}
Quantum mechanics approaches classical mechanics when quantum numbers become
sufficiently high. This occurs at relatively high $T$ for a given quantum
quantity. This means for many purposes we can begin our derivation from a
classical mechanics standpoint and above a certain threshold quantity we achieve
accurate results. For instance the translation quantum number at room
temperature, $n\approx 10^8$, thus a classical mechanical approximation is very
accurate. However, at the same conditions the vibrational quantum number is too
low to achieve accurate results using a classical approach.

\subsection{Molecular Partition Function}
We know for a quantum mechanical derivation of molecular partition functions we
arrive at
\begin{equation*}
	q_p = \sum_{j\in J}{e^{-\beta\H_p}},
\end{equation*}
for a particular separable mode of energy $p$.  Hence, we assume that for a
classical system we take the argument of the summation and integrate over
accessible energy levels,
\begin{equation*}
	q_{p, class} = \int_{\H_p}{e^{-\beta\H_p}\d{\H_p}}.
\end{equation*}
The whole molecular partition function is then,
\begin{align*}
	q_{class} &= \int_{\H}{{e^{-\beta\H}\d{\H}}}\\
			  &= \nint e^{-\beta\H(\vec{p},\vec{q})}
			  \d{\vec{p}}\d{\vec{q}},
\end{align*}
where the second line takes the Hamiltonian to be a function of coordinates
$\vec{q}$ and conjugate momenta $\vec{p}$. The dimensionality of the vectors $q$
and $p$ is simply the dimensionality of the molecule or the number of degrees of
freedom that must be satisfied to fully describe the molecule.

\subsubsection{Translational Partition Function}
We know classically the energy of translation is
\begin{equation*}
	\H = \frac{1}{2m}(p_x^2 + p_y^2 + p_z^2).
\end{equation*}
Then, the translational partition function is,
\begin{align*}
	q_{trans} &= \nint \exp\left(-\frac{\beta(p_x^2 + p_y^2 +
	p_z^2)}{2m}\right)\d{p_x}\d{p_y}\d{p_z}\d{x}\d{y}\d{z}\\
			  &= V \iiint \exp\left(-\frac{\beta(p_x^2 + p_y^2 +
	p_z^2)}{2m}\right)\d{p_x}\d{p_y}\d{p_z}\\
			  &= V \left(\int_{-\infty}^{\infty}{e^{-\frac{\beta p^2}
			  {2m}}}\d{p}\right)^{3}\\
			  &= V \cdot \left(\sqrt{\frac{2m\pi}{\beta}}\right)^{3}\\
			  &= V (2\pi mkT )^{3/2}.
\end{align*}
$V$ comes from the lack of spacial dependence so the volume integral evaluates
to the volume. The last two lines come from the error function. When we compare
this to the classical approximation made in chapter 5,
\begin{equation*}
	V \left(\frac{2\pi mkT}{h^2} \right)^{3/2},
\end{equation*}
we find that a factor of $h^{-3}$ is missing.

\subsubsection{Rotational Partition Function}
For a classical rigid rotor the Hamiltonian is,
\begin{equation*}
	\H = \frac{1}{2I}\left(p_{\theta}^2 +
	\frac{p_{\phi}^{2}}{\sin^{2}\theta}\right).
\end{equation*}
The partition function is then,
\begin{align*}
	q_{rot} &= \int_{-\infty}^{\infty}{} \int_{-\infty}^{\infty}{}
	\int_{0}^{2\pi}{} \int_{0}^{\pi}{e^{-\beta\H}
	\d{p_{\theta}}\d{p_{\phi}}\d{\phi}\d{\theta}}\\
			&= 2\pi \int_{-\infty}^{\infty}{e^{-\beta
			p_{\theta}^{2}/2I}\d{p_{\theta}}}
			\int_{-\infty}^{\infty}{} \int_{0}^{\pi}{e^{-\beta
			p_{\phi}/\sin^{2}\theta}\d{p_{\phi}}\d{\theta}}.
\end{align*}
Note that the first argument of the integral is another error function so we can
further simplify to,
\begin{equation*}
	q_{rot} = 2\pi \cdot \sqrt{2\pi IkT} \int_{-\infty}^{\infty}{}
	\int_{0}^{\pi}{e^{-\beta p_{\phi}/\sin^{2}\theta}\d{p_{\phi}}\d{\theta}}.
\end{equation*}
From here the last two integrals must be carried out. I am not currently sure
how to do them though. The answer becomes,
\begin{equation*}
	q_{rot} = 8\pi^{2} IkT,
\end{equation*}
however. Again notice that a factor of $h^{-2}$ is missing from the partition
function.

\subsubsection{Vibrational Partition Function}
For a harmonic oscillator the Hamiltonian is,
\begin{equation*}
	\H = \frac{p^{2}}{2\mu} + \frac{k}{2}x^{2},
\end{equation*}
where the first term captures the kinetic energy and the second potential. Then,
$q_{vib}$ is,
\begin{align*}
	q_{vib} &= \int_{-\infty}^{\infty}{} \int_{-\infty}^{\infty}{\exp \left(
	-\beta\frac{p^{2}}{2\mu} + \frac{k}{2}x^{2}\right) \d{x}\d{p}}\\
			&= \int_{-\infty}^{\infty}{e^{-\beta p^{2}/2\mu}\d{p}}
			\int_{-\infty}^{\infty}{e^{-\beta kx^{2}/2}\d{x}}\\
			&= (2\mu\pi k_b T)^{1/2}\left(\frac{2\pi k_b T}{k}\right)^{1/2}\\
			&= \frac{k_b T}{\nu} \text{, where } \nu =
			\left(\frac{k}{4\pi^{2}\mu}\right)^{1/2}.
\end{align*}
Note once again this integral was done using the error function and that this is
off by a factor of $h$.

\subsubsection{Adding Planck's Constant}
For each of the molecular partition functions we have examined thus far, the
final result is off by a factor of $h^{s}$ where $s$ is the dimension of
$\vec{p}$ or $\vec{q}$. From this we hypothesis,
\begin{equation*}
	q = \sum_{j\in J}{e^{-\beta\H}} \to \frac{1}{h^{s}}
	\nint{e^{-\beta\H} \prod_{j=1}^{s}{\d{p_{j}}\d{q_{j}}}}.
\end{equation*}

\subsection{Ensemble Partition Function}
Taking the high temperature limit of the ensemble partition function for
noniteracting molecules,
\begin{equation*}
	\Q = \frac{q^{N}}{N!},
\end{equation*}
we apply the classical molecular partition function to arrive at,
\begin{align*}
	\Q &= \frac{1}{N!}\prod_{j=1}^{N}\left({\frac{1}{h^{s}}\nint
			e^{-\beta\H_{j}} \prod_{i=1}^{s}{\d{p_{ji}}\d{q_{ji}}}}\right).\\
	   &= \frac{1}{h^{sN}N!}\prod_{j=1}^{N}{\left(\nint
	   e^{-\beta\H_{j}} \prod_{i=1}^{s}{\d{p_{ji}}\d{q_{ji}}}\right)}.
\end{align*}
To produce the standard formulation of the classical partition function, we
relabel the indices $ji \to k$ so that $k = (j-1)s + i$. That is $k \in [0,s]$
represents the former indices $1i$ or the position and momenta of the first
particle. Doing this allows us to simplify the equation as,
\begin{align*}
	\Q &= \frac{1}{h^{sN}N!}\nint{e^{-\beta\sum_{j}{\H_{j}}}
	\prod_{k=1}^{sN}{\d{p_{k}}\d{q_{k}}}}\\
	   &= \frac{1}{h^{sN}N!}\nint{e^{-\beta\H}
	\prod_{k=1}^{sN}{\d{p_{k}}\d{q_{k}}}}.
\end{align*}
The Hamiltonian in the last line is the many body Hamiltonian not to be confused
with a molecular or sub-molecular Hamiltonian. This is the correct form of the
classical limit even though we assumed noninteracting molecules and merely
postulated the form from an observation of the quantum derivation.

Examining a monatomic gas with a Hamiltonian of,
\begin{equation*}
	\H = \frac{1}{2m} \sum_{j=1}^{N}{(p_{x}^{2} + p^{2}_y + p^{2}_{z}) +
	U(\vec{r}_{1}, \vec{r}_{2}, \ldots, \vec{r}_N )},
\end{equation*}
where $\vec{r}_i$ is the position vector of particle $i$. Then taking the
momentum integrals (using the error function), we get
\begin{equation*}
	\Q_{class} = \frac{1}{N!}\left(\frac{2\pi m kT}{h^{2}}\right)^{3N/2} Z_{N},
\end{equation*}
where $Z_N$ is 
\begin{equation*}
	Z_N = \int_{V}{e^{-\beta U(\vec{r}_{1}, \vec{r}_{2}, \dots, \vec{r}_N )}
	\d{\vec{r}_{1}}\d{\vec{r}_{2}}\cdots\d{\vec{r}_{N}}}.
\end{equation*}
$Z_N$ has a special name, \textit{the classical configuration integral}. This
integral encompasses all the intermolecular interactions present in a system.

\subsection{Hybrid Partition Functions}
When some degrees of freedom can be treated classically and others must be
treated quantumly, if the Hamiltonian is separable in those degrees of freedom,
so is the partition function into a classical and quantum part.
\begin{align*}
	\H &= \H_{class} + \H_{quant}\\
	\Q &= \Q_{class} \Q_{quant}\\
	   &= \frac{\Q_{quant}}{h^{sN}N!}
	   \int{e^{-\beta\H_{class}}\d{\vec{p}_{class}}\d{\vec{q}_{class}}}.
\end{align*}
Similarly, if the molecular Hamiltonian is separable we have,
\begin{align*}
	\H &= \H_{class} + \H_{quant}\\
	q &= q_{class} q_{quant}\\
	   &= \frac{q_{quant}}{h^{s}}
	   \int{e^{-\beta\H_{class}}\d{\vec{p}_{class}}\d{\vec{q}_{class}}}.
\end{align*}

\section{Phase Space and the Liouville Equation}%
\label{sec:psle}

\subsection{Defining of Phase Space and Density of States}
We take a system of $N$ particles, where each particle has $s$ degrees of
freedom, that is $s$ coordinates can uniquely define the particle. To complete
the description of the classical system we also need to define $s$ conjugate
momenta that correspond to the $s$ coordinates. The system is then uniquely
defined by $l=sN$ coordinates and conjugate momenta. The total system then
consists of  $2l$ values.

Having an uniquely defined classical system, we now have motivation to define a
phase space that projects the classical system into a new dynamical space which
we will call the phase space. We project the classical system to a subset of
$\mathbb{R}^{2l}$ which works to a prism of $2l$ dimensional Euclidean space
where the extent of each dimension is the physical limits of the system.  Each
system at a specified time is then a point in this space. This space is the
\textit{phase space} and those points are \textit{phase points}.

For a given phase point, its past and future is uniquely defined by the laws of
motion and the current position in phase space. We postulate that trajectories
are continuous which for a Hamiltonian system is true since
\begin{equation*}
	\dot{q}_{j} = \frac{\partial\H}{\partial p_{j}} \text{ and } \dot{p}_{j} = -
	\frac{\partial\H}{\partial q_{j}} \text{ for } j \in 2l.
\end{equation*}
Thus, trajectories in this frame work are uniquely defined just like phase
points. This means that the evolution of $q$ and $p$ can be defined as some
function of $t$.

We now show the preservation of density on a particular subset of the phase
space. Using the microcanonical ensemble conservation of $N, V, E$, we define a
mapping $M:P\to\Omega$, where $P$ is the phase space and $\Omega$ is the subset
defined by $\{(x_{i})_{1}^{2l} : (x_{i})_{1}^{2l} \sim N, V, E\}$ that is the set
of points that correspond to a particular $N,V,E$. This is an energy ``surface''
of a specified number and volume. We let all $X\in\Omega$ be occupied.  That is
all states are equally likely, this is the classical meaning of the
\textit{postulate of equal a priori probabilities}. This also suggests that
each subdomain of $\Omega$ is equally dense since if this was not the case then
particular regions of the microcanonical ensemble would be favored. However, it
is important to note, that this \textbf{does not imply uniform density in $P$}.

\subsection{Deriving the Louiville Equation}
To begin with deriving the Louivile equation, we first create a function $f$
that gives the infinitesimal density around a phase point. We then have
\begin{equation*}
	\rho = f(p(t), q(t))\d{p}\d{q}.
\end{equation*}
If we integrate $f$ over the entire domain we retrieve $\mathcal{A}$,
\begin{equation*}
	\mathcal{A} = \nint{f(p(t),q(t))\d{p}\d{q}}.
\end{equation*}
Any property averaged over all systems is
\begin{align*}
	\overline{\phi} &= \frac{\nint{\phi(p(t),q(t))f(p(t),q(t))\d{p}\d{q}}}
	{\nint{f(p(t),q(t))\d{p}\d{q}}}\\
	~\\
	\overline{\phi} &= \frac{\nint{\phi(p(t),q(t))f(p(t),q(t))\d{p}\d{q}}}
	{\mathcal{A}}.
\end{align*}
Gibb's postulate was originally formulated by saying this average corresponds to
thermodynamical observed variables.

We now turn our attention to the flux of phase points into and out of an
arbitrary volume about a point $X\in P$. We will prefix quantities in this volume
with $\delta$. The instantaneous number of phase points in $\delta V$ is given
by,
\begin{equation*}
	\delta N = f(p(t),q(t))\delta p_1 \delta q_{1} \delta p_{2} \cdots \delta
	q_{l} \delta p_{l}.
\end{equation*}
This is just taking $f$ at $X$ and smearing it over the volume $\delta V$. To
find the total flux we have to consider $2l$ dimensions. However, each dimension
is equivalent in phase space, i.e.\ momenta and position are not treated
differently. Therefore, we look at the dimension $q_{1}$ and generalize the
results. The net flux in the $q_{1}$ dimension is given by the difference of the
individual fluxes through the orthogonal hyper-surfaces defined by $q_{1}$ and
$q_{1} + \delta q_{1}$. The flux through first face is given by,
\begin{equation*}
	f(q_{1},\ldots,q_{l}, p_{1},\ldots,p_{l})\dot{q}_1 \delta p_1 \cdots \delta p_{l}
	\delta q_1 \cdots \delta q_{l}.
\end{equation*}
Mathematically, this is taking the number of particles per $\delta q_{1}$
or a linear density as can be seen by,
\begin{equation*}
	\frac{\delta N}{\delta q_{1}} = f(q_{1},\ldots,q_{l}, p_{1},\ldots,p_{l})
	\delta p_1 \cdots \delta p_{l} \delta q_2 \cdots \delta q_{l},
\end{equation*}
and multiplying it by the ``velocity'' in the normal direction to the face
$\partial q_{1}/\partial t$. This is analogous to a flux taken in real space.
Likewise, the flux at the face defined by $q_{1} + \delta q_{1}$ is given by,
\begin{equation*}
	f(q_{1} + \delta q_{1},\ldots,q_{l}, p_{1},\ldots,p_{l})
	\dot{q}_{1}(q_{1} + \delta q_{1},q_{2},\ldots,q_{l}, p_{1},\ldots,p_{l})
	\delta p_1 \cdots \delta p_{l} \delta q_2 \cdots \delta q_{l}.
\end{equation*}
Taking a Taylor expansion around $q_{1}$ we find the flux at $\delta q_{1}$ is
approximately,
\begin{equation*}
	\left(f_{q_{1}} + \left(\frac{\partial f}{\partial q_{1}}\right)_{q_{1}}
	\delta q_{1}\right)
	\left((\dot{q}_{1})_{q_{1}} + \left(\frac{\partial \dot{q}_{1}}
	{\partial q_{1}}\right)_{q_{1}}\right)
	\delta p_1 \cdots \delta p_{l} \delta q_2 \cdots \delta q_{l}
\end{equation*}
where we have discarded all higher order terms. If we expand out the terms in
the parenthesis we have,
\begin{equation*}
	f_{q_{1}}(\dot{q}_{1})_{q_{1}}
	+
	\left(\frac{\partial f}{\partial q_{1}}\right)_{q_{1}}(\dot{q}_{1})_{q_{1}}
	+
	f_{q_{1}}\left(\frac{\partial \dot{q}_{1}}{\partial q_{1}}\right)_{q_{1}}
	\delta q_{1}
	+
	\left(\frac{\partial f}{\partial q_{1}}\right)_{q_{1}}
	\left(\frac{\partial \dot{q}_{1}}{\partial q_{1}}\right)_{q_{1}}\delta
	q_{1}^{2}.
\end{equation*}
Noting again that we will ignore all high order terms, $\delta q_{1}^{2}$ is
sufficiently small to ignore it. Then if we take the difference between the
fluxes at the two faces we have,
\begin{equation*}
	-\left(f_{q_{1}}\left(\frac{\partial \dot{q}_{1}}
	{\partial q_{1}}\right)_{q_{1}} + (\dot{q}_{1})_{q_{1}}
	\left(\frac{\partial f}{\partial q_{1}}\right)_{q_{1}}\right)
	\delta p_1 \cdots \delta p_{l} \delta q_1 \cdots \delta q_{l}.
\end{equation*}
Likewise, for $p_{1}$
\begin{equation*}
	-\left(f_{p_{1}}\left(\frac{\partial \dot{p}_{1}}
	{\partial p_{1}}\right)_{p_{1}} + (\dot{p}_{1})_{p_{1}}
	\left(\frac{\partial f}{\partial p_{1}}\right)_{p_{1}}\right)
	\delta p_1 \cdots \delta p_{l} \delta p_1 \cdots \delta p_{l}.
\end{equation*}
Here, we recall that there are a total of $2l$ dimensions to analyze the flux
over $l$ for momentum and $l$ for position. The total change in number in a
instance of time is,
\begin{equation*}
	-\sum_{i=1}^{l}{\left(f_{p_{1}}\left(\frac{\partial \dot{p}_{1}} {\partial
	p_{1}}\right)_{p_{1}} + (\dot{p}_{1})_{p_{1}} \left(\frac{\partial
	f}{\partial p_{1}}\right)_{p_{1}} + f_{q_{1}}\left(\frac{\partial \dot{q}_{1}}
	{\partial q_{1}}\right)_{q_{1}} + (\dot{q}_{1})_{q_{1}}
	\left(\frac{\partial f}{\partial q_{1}}\right)_{q_{1}}\right)
	\delta p_1 \cdots \delta p_{l} \delta p_1 \cdots \delta p_{l}}.
\end{equation*}
The expression is equivalent to,
\begin{equation*}
	\frac{\partial\delta N}{\partial t}.
\end{equation*}
If we divide by the volume element we get the partial derivative of the  density
$f$ w.r.t.\ time. We can further simplify this by noting that 
\begin{equation*}
	\dot{q}_{j} = \frac{\partial\H}{\partial p_{j}} \text{ and } \dot{p}_{j} = -
	\frac{\partial\H}{\partial q_{j}}.
\end{equation*}
This means that the derivatives of $\dot{q}_{i}$ and $\dot{p}_{i}$ are mixed
partials of the Hamiltonian. Since mixed partials are equivalent and the two
differ in signs, they cancel leaving,
\begin{equation*}
	\frac{\partial f}{\partial t} = -\sum_{i=1}^{l}{\left((\dot{p}_{1})_{p_{1}}
	\left(\frac{\partial f}{\partial p_{1}}\right)_{p_{1}} +
	(\dot{q}_{1})_{q_{1}} \left(\frac{\partial f}{\partial
	q_{1}}\right)_{q_{1}}\right)}.
\end{equation*}
This is the Louville equation which looks very much like a statement of the
conservation of mass because it is the equivalent in phase space. Another common
form of the Louville equation given in terms of the Hamiltonian is,
\begin{equation*}
	\frac{\partial f}{\partial t} = -
	\sum_{i=1}^{l}{\left(\frac{\partial\H}{\partial p_{j}} \frac{\partial
		f}{\partial q_{j}} - \frac{\partial\H}{\partial q_{j}} \frac{\partial
		f}{\partial p_{j}}\right)}.
\end{equation*}
The last form just follows from the definition of the partials of the
Hamiltonian.

\subsection{The Conservation of Extension and Incompressibility}
By moving the previous equation to one side we get an expression equal to 0,
\begin{equation*}
	\frac{\partial f}{\partial t} + \sum_{i=1}^{l}{\left((\dot{p}_{1})_{p_{1}}
	\left(\frac{\partial f}{\partial p_{1}}\right)_{p_{1}} +
	(\dot{q}_{1})_{q_{1}} \left(\frac{\partial f}{\partial
	q_{1}}\right)_{q_{1}}\right)} = 0.
\end{equation*}
We should note though as $f = f(\vec{p},\vec{q},t)$ the left hand size is the
total derivative, that is,
\begin{equation*}
	\frac{\d{f}}{\d{t}} = 0.
\end{equation*}
To put this into words, the local environment of a phase point at any time $t$
compared to an earlier or later time $t_{0}$ is equally dense. Another way to
phrase this is that the phase space behaves as an incompressible fluid.

Furthermore, if we take the volume element around a phase point $X$ at time
$t_{0}$, for time $t_{0} + t$ the density of the volume element must be the
same.  However, no point can cross the boundary of this volume because to do so
two trajectories with different intial conditions would have to converge at the
time $t_{0} + t$. Since each trajectory is unique in phase space, this cannot
happen. Since both number and density of the volume element are conserved, then,
volume must be conserved as well. This property is called \textit{conservation
of extension in phase space}, or,
\begin{equation*}
	\delta \vec{p} \delta \vec{q} = \delta \vec{p}_{0} \delta \vec{q}_{0}.
\end{equation*}
Mathematically this can be shown by taking the Jacobian from $(\vec{p},\vec{q})$
to $(\vec{p}_{0},\vec{q}_{0})$ and proving that is is unity. From this theorem,
an important corollary can be drawn, namely the equivalence of different
coordinate systems. Any two sets of coordinates and conjugate momenta that
describe a system in phase space equally well, have equivalent volume elements.
\begin{align*}
	&\text{if these are two coordiante systems in phase space,}\\
	&q_{1}, q_{2},\ldots,q_{3n},p_{1},p_{2},\ldots,p_{3n}\\
	&Q_{1}, Q_{2},\ldots,Q_{3n},P_{1},P_{2},\ldots,P_{3n}\\
	&\text{then,}\\
	&\d q_{1}, \d q_{2},\ldots,\d q_{3n},\d p_{1},\d p_{2},\ldots,\d p_{3n} =
	\d Q_{1}, \d Q_{2},\ldots,\d Q_{3n},\d P_{1},\d P_{2},\ldots,\d P_{3n}.
\end{align*}

\section{Equipartition of Energy}%
\label{sec:eoe}
In our previous classical analysis of the average energy contributions by
different molecular modes of energy, the energy contribution of classical modes
was always $kT/2$. We now show for all Hamiltonian's that can be written as 
\begin{equation*}
	\H = \sum_{j=1}^{m}{a_{j}p_{j}^{2}} + \sum_{j=1}^{n}{b_{j}q_{j}^{2}} +
	\H(p_{m+1},\ldots, p_{s}, q_{n+1},\ldots,q_{s}),
\end{equation*}
where $a_{j}$ and $ b_{j}$ all quadratic contributions contribute $kT/2 $ to the
energy.

First we note that for the separable Hamiltonian, the partition function is also
separable,
\begin{align*}
	q &= \frac{1}{h^{s}}\nint e^{-\beta(\sum_{j=1}^{m}{\H_{p_{j}}} +
	\sum_{j=1}^{n}{\H_{q_{j}}} + \H_{else})}\\
	  &= \frac{1}{h^{s}}\prod_{j=1}^{m}{%
		  \int{p_{j}}{e^{-\beta \H_{p_{j}}\d{p_{j}}}}%
	  }
		  \prod_{j=1}^{n}{\int_{q_{j}}{e^{-\beta H_{q_{j}}}\d{q_{j}}}}
		  \nint{e^{-\beta \H_{else}} \d{q_{n+1}} \ldots \d{q_{s}} \d{p_{m+1}}
		  \ldots \d{p_{s}}}\\
	  &= \prod_{j=1}^{n}{q_{q_{j}}} \prod_{j=1}^{m}{q_{p_{j}}} \cdot q_{else}.
\end{align*}
Thus, for a particular energy mode in the conjugate momenta $p_{i}~~ 0 < i \le
m$, the average energy in that mode is
\begin{align*}
	\langle \H_{i} \rangle &= \frac{\nint{\H_{i}e^{-\beta\H}}}{q}\\
						   &= \frac{%
							   \prod_{j=1}^{n}{q_{q_{j}}}%
							   \prod_{j=1,~j \ne i}^{m}{q_{p_{j}}}%
							   \cdot q_{else}
						   \int{\H_{i}e^{-\beta\H_{i}}\d{p_{i}}}}
						   {\prod_{j=1}^{n}{q_{q_{j}}}%
							\prod_{j=1}^{m}{q_{p_{j}}}%
					\cdot q_{else}}\\
						   &= \frac{\int{\H_{i}e^{-\beta\H_{i}}\d{p_{i}}}}
						   {q_{p_{i}}}.
\end{align*}
From here we can use integration by parts to determine $\langle \H_{i} \rangle$,
\begin{align*}
	\langle \H_{i} \rangle &= \frac{1}{q_{p_{i}}} \int{a_{j}p_{j}^{2} e^{-\beta
	\H_{i}}}\\
			\text{Let, } &u = -\frac{p_{i} kT}{2} \qquad \d{u} =
			\frac{kT}{2}\d{p_{i}}\\
						 &\d{v} = -\frac{-a_{j}p_{j}}{kT}e^{-\beta
						 p_{j}^{2}/kT}\d{p_{j}} \qquad
						 v = e^{-a_{j}p_{j}^{2}/kT}\\
						 &= \frac{1}{q_{p_{i}}} \left(
							 \left[-\frac{p_{i} kT}{2} e^{-a_{j}p_{j}^{2}/kT}
							 \right]_{\alpha}^{\delta} +
							 \frac{kT}{2} \int_{\alpha}^{\delta}{%
						 e^{-a_{j}p_{j}^{2}/kT}\d{p_{i}}}\right).
\end{align*}
An important aside for momentum $[\alpha, \delta)$ represent $[0,\infty)$, but
for coordinates can vary but some notable examples are  $[0, L]$ or  $[-L/2,
L/2]$ where  $L$ is the length of that dimension. In other words, either $q_{j}
= 0$ or $\H_{j}\to\infty$ at the two limits of integration. The second is due to
the potential at the wall being infinite. Therefore, in both cases of momentum
and position, the limits of integration make the first term above zero. Then it
is easy to see,
\begin{align*}
		\langle \H_{i} \rangle &= \frac{1}{q_{p_{i}}} \left(
						\frac{kT}{2} \int_{\alpha}^{\delta}{%
						e^{-a_{j}p_{j}^{2}/kT}\d{p_{i}}}\right)\\
							   &= \frac{kTq_{p_{i}}}{2q_{p_{i}}}\\
							   &= \frac{kT}{2}.
\end{align*}
This result also holds when $a_{j}$ or $b_{j}$ is a function of a different
variable.
