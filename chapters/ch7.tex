\section{Formulation of the Classical Partition Function}%
\label{sec:cpf}
Quantum mechanics approaches classical mechanics when quantum numbers become
sufficiently high. This occurs at relatively high $T$ for a given quantum
quantity. This means for many purposes we can begin our derivation from a
classical mechanics standpoint and above a certain threshold quantity we achieve
accurate results. For instance the translation quantum number at room
temperature, $n\approx 10^8$, thus a classical mechanical approximation is very
accurate. However, at the same conditions the vibrational quantum number is too
low to achieve accurate results using a classical approach.

\subsection{Molecular Partition Function}
We know for a quantum mechanical derivation of molecular partition functions we
arrive at
\begin{equation*}
	q_p = \sum_{j\in J}{e^{-\beta\H_p}},
\end{equation*}
for a particular separable mode of energy $p$.  Hence, we assume that for a
classical system we take the argument of the summation and integrate over
accessible energy levels,
\begin{equation*}
	q_{p, class} = \int_{\H_p}{e^{-\beta\H_p}\d{\H_p}}.
\end{equation*}
The whole molecular partition function is then,
\begin{align*}
	q_{class} &= \int_{\H}{{e^{-\beta\H}\d{\H}}}\\
			  &= \nint e^{-\beta\H(\vec{p},\vec{q})}
			  \d{\vec{p}}\d{\vec{q}},
\end{align*}
where the second line takes the Hamiltonian to be a function of coordinates
$\vec{q}$ and conjugate momenta $\vec{p}$. The dimensionality of the vectors $q$
and $p$ is simply the dimensionality of the molecule or the number of degrees of
freedom that must be satisfied to fully describe the molecule.

\subsubsection{Translational Partition Function}
We know classically the energy of translation is
\begin{equation*}
	\H = \frac{1}{2m}(p_x^2 + p_y^2 + p_z^2).
\end{equation*}
Then, the translational partition function is,
\begin{align*}
	q_{trans} &= \nint \exp\left(-\frac{\beta(p_x^2 + p_y^2 +
	p_z^2)}{2m}\right)\d{p_x}\d{p_y}\d{p_z}\d{x}\d{y}\d{z}\\
			  &= V \iiint \exp\left(-\frac{\beta(p_x^2 + p_y^2 +
	p_z^2)}{2m}\right)\d{p_x}\d{p_y}\d{p_z}\\
			  &= V \left(\int_{-\infty}^{\infty}{e^{-\frac{\beta p^2}
			  {2m}}}\d{p}\right)^{3}\\
			  &= V \cdot \left(\sqrt{\frac{2m\pi}{\beta}}\right)^{3}\\
			  &= V (2\pi mkT )^{3/2}.
\end{align*}
$V$ comes from the lack of spacial dependence so the volume integral evaluates
to the volume. The last two lines come from the error function. When we compare
this to the classical approximation made in chapter 5,
\begin{equation*}
	V \left(\frac{2\pi mkT}{h^2} \right)^{3/2},
\end{equation*}
we find that a factor of $h^{-3}$ is missing.

\subsubsection{Rotational Partition Function}
For a classical rigid rotor the Hamiltonian is,
\begin{equation*}
	\H = \frac{1}{2I}\left(p_{\theta}^2 +
	\frac{p_{\phi}^{2}}{\sin^{2}\theta}\right).
\end{equation*}
The partition function is then,
\begin{align*}
	q_{rot} &= \int_{-\infty}^{\infty}{} \int_{-\infty}^{\infty}{}
	\int_{0}^{2\pi}{} \int_{0}^{\pi}{e^{-\beta\H}
	\d{p_{\theta}}\d{p_{\phi}}\d{\phi}\d{\theta}}\\
			&= 2\pi \int_{-\infty}^{\infty}{e^{-\beta
			p_{\theta}^{2}/2I}\d{p_{\theta}}}
			\int_{-\infty}^{\infty}{} \int_{0}^{\pi}{e^{-\beta
			p_{\phi}/\sin^{2}\theta}\d{p_{\phi}}\d{\theta}}.
\end{align*}
Note that the first argument of the integral is another error function so we can
further simplify to,
\begin{equation*}
	q_{rot} = 2\pi \cdot \sqrt{2\pi IkT} \int_{-\infty}^{\infty}{}
	\int_{0}^{\pi}{e^{-\beta p_{\phi}/\sin^{2}\theta}\d{p_{\phi}}\d{\theta}}.
\end{equation*}
From here the last two integrals must be carried out. I am not currently sure
how to do them though. The answer becomes,
\begin{equation*}
	q_{rot} = 8\pi^{2} IkT,
\end{equation*}
however. Again notice that a factor of $h^{-2}$ is missing from the partition
function.

\subsubsection{Vibrational Partition Function}
For a harmonic oscillator the Hamiltonian is,
\begin{equation*}
	\H = \frac{p^{2}}{2\mu} + \frac{k}{2}x^{2},
\end{equation*}
where the first term captures the kinetic energy and the second potential. Then,
$q_{vib}$ is,
\begin{align*}
	q_{vib} &= \int_{-\infty}^{\infty}{} \int_{-\infty}^{\infty}{\exp \left(
	-\beta\frac{p^{2}}{2\mu} + \frac{k}{2}x^{2}\right) \d{x}\d{p}}\\
			&= \int_{-\infty}^{\infty}{e^{-\beta p^{2}/2\mu}\d{p}}
			\int_{-\infty}^{\infty}{e^{-\beta kx^{2}/2}\d{x}}\\
			&= (2\mu\pi k_b T)^{1/2}\left(\frac{2\pi k_b T}{k}\right)^{1/2}\\
			&= \frac{k_b T}{\nu} \text{, where } \nu =
			\left(\frac{k}{4\pi^{2}\mu}\right)^{1/2}.
\end{align*}
Note once again this integral was done using the error function and that this is
off by a factor of $h$.

\subsubsection{Adding Planck's Constant}
For each of the molecular partition functions we have examined thus far, the
final result is off by a factor of $h^{s}$ where $s$ is the dimension of
$\vec{p}$ or $\vec{q}$. From this we hypothesis,
\begin{equation*}
	q = \sum_{j\in J}{e^{-\beta\H}} \to \frac{1}{h^{s}}
	\nint{e^{-\beta\H} \prod_{j=1}^{s}{\d{p_{j}}\d{q_{j}}}}.
\end{equation*}

\subsection{Ensemble Partition Function}
Taking the high temperature limit of the ensemble partition function for
noniteracting molecules,
\begin{equation*}
	\Q = \frac{q^{N}}{N!},
\end{equation*}
we apply the classical molecular partition function to arrive at,
\begin{align*}
	\Q &= \frac{1}{N!}\prod_{j=1}^{N}\left({\frac{1}{h^{s}}\nint
			e^{-\beta\H_{j}} \prod_{i=1}^{s}{\d{p_{ji}}\d{q_{ji}}}}\right).\\
	   &= \frac{1}{h^{sN}N!}\prod_{j=1}^{N}{\left(\nint
	   e^{-\beta\H_{j}} \prod_{i=1}^{s}{\d{p_{ji}}\d{q_{ji}}}\right)}.
\end{align*}
To produce the standard formulation of the classical partition function, we
relabel the indices $ji \to k$ so that $k = (j-1)s + i$. That is $k \in [0,s]$
represents the former indices $1i$ or the position and momenta of the first
particle. Doing this allows us to simplify the equation as,
\begin{align*}
	\Q &= \frac{1}{h^{sN}N!}\nint{e^{-\beta\sum_{j}{\H_{j}}}
	\prod_{k=1}^{sN}{\d{p_{k}}\d{q_{k}}}}\\
	   &= \frac{1}{h^{sN}N!}\nint{e^{-\beta\H}
	\prod_{k=1}^{sN}{\d{p_{k}}\d{q_{k}}}}.
\end{align*}
The Hamiltonian in the last line is the many body Hamiltonian not to be confused
with a molecular or sub-molecular Hamiltonian. This is the correct form of the
classical limit even though we assumed noninteracting molecules and merely
postulated the form from an observation of the quantum derivation.

Examining a monatomic gas with a Hamiltonian of,
\begin{equation*}
	\H = \frac{1}{2m} \sum_{j=1}^{N}{(p_{x}^{2} + p^{2}_y + p^{2}_{z}) +
	U(\vec{r}_{1}, \vec{r}_{2}, \ldots, \vec{r}_N )},
\end{equation*}
where $\vec{r}_i$ is the position vector of particle $i$. Then taking the
momentum integrals (using the error function), we get
\begin{equation*}
	\Q_{class} = \frac{1}{N!}\left(\frac{2\pi m kT}{h^{2}}\right)^{3N/2} Z_{N},
\end{equation*}
where $Z_N$ is 
\begin{equation*}
	Z_N = \int_{V}{e^{-\beta U(\vec{r}_{1}, \vec{r}_{2}, \dots, \vec{r}_N )}
	\d{\vec{r}_{1}}\d{\vec{r}_{2}}\cdots\d{\vec{r}_{N}}}.
\end{equation*}
$Z_N$ has a special name, \textit{the classical configuration integral}. This
integral encompasses all the intermolecular interactions present in a system.

\subsection{Hybrid Partition Functions}
When some degrees of freedom can be treated classically and others must be
treated quantumly, if the Hamiltonian is separable in those degrees of freedom,
so is the partition function into a classical and quantum part.
\begin{align*}
	\H &= \H_{class} + \H_{quant}\\
	\Q &= \Q_{class} \Q_{quant}\\
	   &= \frac{\Q_{quant}}{h^{sN}N!}
	   \int{e^{-\beta\H_{class}}\d{\vec{p}_{class}}\d{\vec{q}_{class}}}.
\end{align*}
Similarly, if the molecular Hamiltonian is separable we have,
\begin{align*}
	\H &= \H_{class} + \H_{quant}\\
	q &= q_{class} q_{quant}\\
	   &= \frac{q_{quant}}{h^{s}}
	   \int{e^{-\beta\H_{class}}\d{\vec{p}_{class}}\d{\vec{q}_{class}}}.
\end{align*}

\section{Phase Space and the Liouville Equation}%
\label{sec:psle}

\subsection{Density of States}

\subsection{Deriving the Louiville Equation}

\subsection{The Conservation of Extension and Incompressibility}

\section{Equipartition of Energy}%
\label{sec:eoe}
