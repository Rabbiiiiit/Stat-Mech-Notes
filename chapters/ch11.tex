% Crystals chapter 11

\section{Degrees of Freedom in a Crystal}%
\label{sec:crystal_dof}
The degrees of freedom within a crystal are essentially all vibrations. To show
this take a crystal composed of $N$ particles. Then, there are $3N$ coordinates
that uniquely specify the crystal's configuration. Like the polyatomic molecule,
3 coordinates can describe the center of mass for translational motion, and 3
can describe the rotation of the crystal. Thus, $3N - 6$ describe the vibration
of atoms around their equilibrium positions. Since at large $N$, $3N \approx 3N
- 6$, the translational and rotational degrees of freedom can be safely ignored.

\section{Derivation of the Partition Function}%
\label{sec:crystal_dpf}
In the crystal we assume that the potential well each particle experiences is
steep, so that deviations from the lattice sites are small. Because deviations
are small, a Taylor's series expansion of the potential, $U$, around a deviation
of 0, $\zeta = 0$, can be taken.
\begin{equation*}
	U(\zeta_1 , \zeta_2 , \ldots, \zeta_{N}) = U(0, 0, \ldots, 0) +
	\sum_{j=1}^{N}{{\left(\frac{\partial U}{\partial \zeta_j }\right)}_0 \zeta_j}
	+ \frac{1}{2} \sum_{i=1}^{N} \sum_{j=1}^{N}{{\left(\frac{\partial^2
	U}{\partial \zeta_i \partial \zeta_j }\right)}_0 \zeta_i \zeta_j } + \cdots
\end{equation*}
As the first derivative at a local extrema is 0, this reduces to
\begin{align*}
	U(\zeta_1 , \zeta_2 , \ldots, \zeta_{N}) &\approx U(0, 0, \ldots, 0) +
	\frac{1}{2} \sum_{i=1}^{N} \sum_{j=1}^{N}{{\left(\frac{\partial^2
	U}{\partial \zeta_i \partial \zeta_j }\right)}_0 \zeta_i \zeta_j }\\
	&\approx U(0, 0, \ldots, 0) + \frac{1}{2} \sum_{i,j}^{N}{k_{ij} \zeta_i
	\zeta_{j}},
\end{align*}
where $k_{ij}$ is the force constant for all pairs of particles. We can see then
that $U(\vec{\zeta})$ is a quadratic function. In addition, $U(\vec{0})$
is not a function of displacement, but is a function of density. This occurs
because the interatomic potential changes when the lattice spacing changes. One
way to show this is by writing $U(\vec{0})$ as $U(\vec{0}; \rho)$. This system
of equations is coupled, meaning the  $i$-th and $j$-th terms are inseparable.
However, just like the polyatomic case, the vibrational modes can be separated
using normal coordinate analysis into $3N - 6$ independent harmonic oscillators.
Taking the equation from Chapter~\ref{ch7}, 
\begin{equation*}
	\nu_j = \frac{1}{2\pi} {\left( \frac{k_j }{\mu_j }\right)}^{1/2}, 
\end{equation*}
we can generate a partition function ignoring the fact we do not know $k$ or
$\mu$.
\begin{equation*}
	\Q(V/N, T) = e^{-U(\vec{0}; \rho)/kT} \prod_{j=1}^{3N - 6}{q_{vib, j}}
\end{equation*}
Q is a function of the inverse of density because $k_{j}$ is related to $k_{ij}$
from the Taylor expansion, and the potential depends on the lattice spacing as
does $U(\vec{0}; \rho)$. We have ignored the translational and rotational
contribution since these are nearly meaningless in this case. Since the
particles are restricted to a specific lattice site, the particles are in fact
distinguishable and no factor of $\frac{1}{N!}$ needs to be introduced. $\Q$ can
also be expanded as so,
\begin{equation*}
	\Q = \prod_{j=1}^{3N}{{\left(\frac{e^{-h\nu_{j}/2kT}}{1 -
	e^{-h\nu_{j}/kT}}\right)}e^{-U(\vec{0}; \rho)/kT}}.
\end{equation*}
Taking the natural log,
\begin{equation*}
	\ln{\Q} = - \frac{-U(\vec{0}; \rho)}{kT} +
	\sum{j=1}^{3N}{{\left(\frac{-h\nu_{j}}{2kT} - \ln{1 -
	e^{-h\nu_{j}/kT}}\right)}}.
\end{equation*}
As $3N$ is large, the distribution of frequencies can be taken to be continuous.
That means the summation can be substituted for an integral with a density of
frequencies term added to ensure a correct weighting of frequencies.
\begin{align*}
	\ln{\Q} &= - \frac{-U(\vec{0}; \rho)}{kT} +
	\int_{0}^{\infty}{{\left(\frac{-h\nu}{2kT} - \ln{1 -
	e^{-h\nu/kT}}\right)} \cdot g(\nu) \d{\nu}}\\
	\int_{0}^{\infty}{g(\nu)\d{\nu}} &= 3N.
\end{align*}
Thus if $g(\nu)$ is known, all thermodynamic properties of the crystal can be
calculated. The specific heat formula is given to compare the Einstein, Debye,
and phonon methods of looking at crystals,
\begin{equation*}
	C_v = k \int_{0}^{\infty}{\frac{(h\nu /kT)^2 e^{-h\nu/kT} g(\nu) \d{\nu}}{(1
	- e^{-h\nu/kT})^2}}.
\end{equation*}

\section{Einstein Crystal}%
\label{sec:crystal_ein}
The Einstein model of crystals assumes that all particles vibrate independently
from each other and have identical local environments. This means they all
vibrate at the same frequency $\nu$. Classically this would just lead to a
constant heat capacity of $3Nk = 3R = 6 cal/deg \cdot mole$. In fact, this
relation is known as the law of Dulong and Petit and comes from the law of
equipartition which states that each quadratic degree of freedom contributes $k
cal/deg \cdot mole$ per atom. However, the D\&P law only holds asymptotically at
sufficiently high temperatures. At lower temperatures $C_v$ drops as $T^3$.
What, Einstein did that was novel and provided an advance was to quantized the
vibrational modes first which is of course correct since all energy is
quantized. 

For an Einstein crystal $g(\nu)$ is given by $3N \delta(\nu - \nu_{E})$, where
$\nu_E$ is the Einstein frequency of a crystal. The integral of the delta
function is just the remainder of the integrand evaluated at $\nu_E$. Then, 
 \begin{equation*}
	 \ln{\Q} = - \frac{-U(\vec{0}; \rho)}{kT} + {\left(\frac{-h\nu_E}{2kT} -
			 \ln{1 - e^{-h\nu_E/kT}}\right)},
\end{equation*}

and the specific heat is,
\begin{align*}
	C_v &= 3Nk {\left(\frac{h\nu_{E}}{kT}\right)}^2 \frac{e^{-h\nu_{E}/kT}}{(1 -
	e^{-h\nu_E/kT})^2} \\
		&= 3Nk {\left(\frac{\Theta_E}{T}\right)}^2 \frac{e^{-\Theta_E /T}}{(1 -
		e^{-\Theta_E /T})^2}.
\end{align*}
In the last equation, $\Theta_E = h\nu_E / k$ and is known as the Einstein
temperature.

While Einstein's theory provided must further understanding about crystal
behavior, it incorrectly predicts the low temperature behavior. The low
temperature limit is,
\begin{equation*}
	C_v \to 3Nk {\left(\frac{\Theta_{E}}{T}\right)}^2 e^{-\Theta_E /T}.
\end{equation*}

\section{Debye Crystal}%
\label{sec:crystal_debye}
The Debye view of a crystal assumes the entire crystal behaves as an elastic
continuous body. This assumption allows for accurate treatment of the low
frequency, large wavelength regime of a crystal's normal frequencies. In the
limit of infinitely long wavelength's this approximation is exact. To analyze
the frequencies then we must define the wave travelling through the continuum
crystal. A wave function can be written as
 \begin{equation*}
	 u(r, t) = A e^{i(\vec{k} \cdot \vec{r} - \omega t)},
\end{equation*}
with wave vector $ \vec{k}$ of magnitude $\frac{2\pi}{\lambda}$ and frequency
$\omega = 2\pi\nu$. The wave vector defines a direction and the frequency
defines how quickly the phase changes. By adding another wave in direction $-k$,
we can form a standing wave,
\begin{align*}
	u &= A( e^{i(\vec{k} \vec{r} - \omega t)} + e^{i(-\vec{k} \vec{r} - \omega
	t)})\\
	  &= 2A e^{i \vec{k} \vec{r}} \cos{\omega t}.
\end{align*}
We need the imaginary component to vanish at the crstal edges to ensure the wave
is standing. This can be accomplished by setting $k_{xi} L_{xi} = n_{xi} \pi$,
or,
\begin{equation*}
	\vec{k} = \frac{\pi}{L} \vec{n}.
\end{equation*}
This just ensures that in every direction the vibration of the endpoints of the
crystal go to zero. The magnitude of k is given by
\begin{equation*}
	|\vec{k}| = {\left[{\left(\frac{\pi}{L}\right)}(n_{x}^2 + n_{y}^2 +
	n_{z})^2\right]}^{1/2}.
\end{equation*}
Frequency is just $\nu = \omega / |\vec{k}|$. The density of wave numbers can be
taken in the same manner as the density of translational states of a molecule,
so
\begin{align*}
	R^2 &= {\left(\frac{kL}{\pi}\right)}^2 = (n_{x}^2 + n_{y}^2 + n_{z})^2\\
	\phi(k) &= \frac{1}{8} \frac{4}{3} \pi R^3 = \frac{\pi}{6}
	{\left(\frac{kL}{\pi}\right)}^3 = \frac{k^3 L^3}{6 \pi^2}.
\end{align*}
The method here takes the volume of one octant of a sphere of radius $R$. The
density between $k$ and $k +\d{k}$ is then,
\begin{equation*}
	\delta(k) \d{k} = \frac{\d{\phi}}{\d{k}} \d{k} = \frac{L^3 k^2 \d{k}}{2
	\pi^2}.
\end{equation*}
Using $\nu = \upsilon / \lambda$ where $\upsilon$ is the velocity of
propagation, and $\nu = \upsilon k /2\pi$ we have,
\begin{equation*}
	g(\nu) \d{\nu} = \frac{4\pi L^3 \nu^2}{\upsilon^3} \d{\nu}.
\end{equation*}

For every magnitude of $\vec{k}$, we can create two linearly independent
transversal waves and one longitudinal wave. The speed of propagation for these
two types of waves can be different and must be averaged. The final result is
then,
\begin{align*}
	g(\nu) \d{\nu} &= 4\pi L^3 \nu^2 {\left(\frac{2}{\upsilon_t^3} +
	\frac{1}{\upsilon_{l}^3}\right)} \d{\nu}\\
				   &= \frac{12\pi L^3 \nu^2}{\upsilon_{0}^3}.
\end{align*}
In the last line $\upsilon_0$ represents the effective velocity of propagation
in the medium. To ensure the proper nomralization of $g(\nu)$ a frequency
$\nu_D$ is defined, called the Debye frequency, such that
\begin{equation*}
	\int_{0}^{\nu_D}{g(\nu)\d{\nu}} = 3N.
\end{equation*}
In terms of other quantities this makes $\nu_{D}$,
\begin{equation*}
	\nu_D = {\left(\frac{3N}{4\pi L^3}\right)}^{1/3} \upsilon_0.
\end{equation*}
Specific heat in the theory is then,
\begin{align*}
	C_v &= 9Nk{\left(\frac{T}{\Theta_{D}}\right)}^3 \int_{0}^{\Theta_D
	/T}{\frac{x^4 e^x}{(e^{x} - 1)^2} \d{x}}\\
		\Theta_D &= \frac{h \nu_D}{k}.
\end{align*}

\subsubsection{Correspondence of States}
We should note that in both the case for Einstein's and Debye's theories that
heat capacity ends up being a function of temperature ratioed with some material
specific constant. This means that if the curves for heat capacity is normalized
with respect to $\Theta_{D}$ or $ \Theta_E$then all the curves are identical.
This implies that the behavior of all crystals once normalized is identical. The
general name for a result like this is the \textit{theory of corresponding
states}. The theory is true for all phases of matter not just crystals that are
guided by the same driving forces.

At low temperatures $(\Theta_D / T) \to \infty$, and the integral converges to
$4\pi^4 / 15$. The low temperature limit in Debye theory is then,
\begin{equation*}
	C_v = \frac{12\pi^4}{5} Nk {\left(\frac{T}{\Theta_{D}}\right)}^3.
\end{equation*}
This is the famous $T^3$-law that was experimentally seen for crystals at low
temperatures.

It bears saying that though Debye's theory has the correct limits; it is still
approximate and $\Theta_D$ which should be constant across $T$ often varies as a
function of $T$. Furthermore, since Debye theory is a continuum theory the
spectra of diatomic crystals is not readily explained by it. Another interesting
point is that the thermodynamic properties of crystals can be estimated from
their mechanical, i.e. elastic, properties.

\section{The Phonon Gas Model}%
\label{sec:crystal_phonon}
Yet another way to derive the properties of a crystal is to treat it as an ideal
gas of phonons. To derive what a phonon is we first reconsider the energy of a
crystal in terms of its normal coordinates,
\begin{align*}
	E_{cryst} &= \sum_{j=1}^{3N}{h\nu_j (n_j + \frac{1}{2}})\\
			  &= \sum_{j=1}^{3N}{h\nu_j n_j} +
			  \sum_{j=1}^{3N}{\frac{h\nu_j}{2}}\\
	E_{cryst}(\{n_{j}\}) &= \sum_{j=1}^{3N}{h\nu_j n_j} + E_0.
\end{align*}
The last line explicitly shows that the energy is a function of the ``occupancy
number'' of the particular mode of energy. We should note that the occupancy
number of a particle mode is not restricted, and, equally important, the sum over
occupancy numbers is not conserved. In this formulation, each of the $3N$
vibrational modes is a phonon which acts as an ideal gas of bosons. The choice
of treating phonons as bosons comes from the fact that phonons do not have
occupancy restrictions.

We can mathematically write down the expected occupancy of a phonon using
Bose-Einstein statistics,
\begin{equation*}
	\overline{n}_j = \frac{1}{\lambda^{-1} e^{\beta \varepsilon_{j}} - 1}.
\end{equation*}
However, we do not immediately know $\lambda$. We do know that the number is not
conserved. This requires that $\mu=0$ or $\lambda=1$ in these cases.
Thermodynamically, however, the requirement must be $\mu = 0$. If we treat the
transition of occupancy number as a reaction, following
from the notes on chemical equilibrium for the ``reaction'' $nA \leftrightarrow
mA$ $n \ne m$, $(m-n)\mu = 0$, and $\mu=0$ and $\lambda = 1$. Thus,
\begin{equation*}
	\overline{n}_j = \frac{1}{e^{\beta \varepsilon_{j}} - 1}.
\end{equation*}
The average energy is then,
 \begin{equation*}
	 \overline{E} = \sum_{j=1}^{3N}{\frac{h\nu_j n_j}{e^{\beta \varepsilon_{j}}
	 - 1}} + E_0.
\end{equation*}
Using the function $g(\nu)$ as before,
\begin{equation*}
	\overline{E} = E_0 + \int_{0}^{\infty}{\frac{g(\nu)h\nu}{e^{\beta
	h\nu} - 1} \d{\nu}}.
\end{equation*}
This equation is equivalent to the previous treatment in the beginning of this
chapter. Thus a fully examination of the statistical mechanics of crystal can be
done from the perspective of phonons. Phonons also have useful quirks. Since
they are pseudo-particles, phonons can have momentum and the transfer between
phonons can be examined. This transfer is also the cause for electrical
resistance in metals. In addition, the interaction of phonons with photons
can be studied and is the basis of Brillouin scattering and many crystal
observed optical effects.

\section{Lattice Dynamics}%
\label{sec:crystals_ld}
Lattice dynamics seek to exactly determine the vibrational frequencies of a
lattice. To examine this technique two 1 dimensional lattices will be examine
the monoatomic and diatomic cases.

\subsection{Monotonic 1D Crystal}
The Hamiltonian following from the beginning of this chapter is
\begin{equation*}
	\H = \sum_{j=1}^{N}{\frac{m}{2}\zeta_{j}^2} +  \sum_{j=2}^{N}{\frac{f}{2}
	{(\zeta_j - \zeta_{j-1})}^2} = K + U,
\end{equation*}
where $f$ are the force constants. The equations of motion for this system is
\begin{equation*}
	m\ddot{\zeta}_j = f(\zeta_{j+1} + \zeta_{j-1} - 2\zeta_{j}).
\end{equation*}
This is simple stating that $ma = F$. We can further assume since the
oscillation is harmonic that
\begin{equation*}
	\zeta_j (t) = e^{i\omega t} y_j,
\end{equation*}
where $y_j$ represents the spacial component of the oscillation. This allows
for a simplification of the equations of motion to,
\begin{equation*}
	-m \omega^2 y_j = f(y_{j+1} + y_{j-1} - 2y_{j}).
\end{equation*}
This difference equation, which mimics differential equations, is of the form
$e^{i\phi}$. Then,
\begin{align*}
	-m \omega^2 &= f(e^{1i \phi} + e^{-1i \phi} - 2 e^{0i \phi})\\
				&= f(2 \cos{\phi} - 2).
\end{align*}
Using the relation $\sin^2 \theta + \cos^2 \theta = 1 $,
\begin{equation*}
	\omega^2 = \frac{4f}{m} \sin^2{\frac{\phi}{2}}.
\end{equation*}
Because $\zeta$ is harmonic it repeats every $\Delta = 2\pi / \phi$. This
implies that $\lambda = 2\pi a / \phi$ where $a$ is the lattice spacing. Thus,
$\phi = 2\pi a / \lambda = ka$. Taking this all together we can get the
dispersion function which shows how different waves travel at different
velocities in a crystal.
\begin{align*}
	\omega^2 &= \frac{4f}{m} \sin^2{\frac{\phi}{2}}\\
	\omega &= {\left(\frac{4f}{m}\right)}^{1/2} | \sin{\frac{\phi}{2}} |
	= \omega_{max} | \sin{\frac{\phi}{2}} |\\
		   &= \omega_{max} | \sin{\frac{ka}{2}} |\\
	\lambda \nu &= \frac{\omega_{max}}{k}| \sin{\frac{ka}{2}} | = c(k).
\end{align*}
Note that $\omega_{max}$ is the maximum frequency in the crystal. As $\lambda
\nu$ is usually not constant, different waves travel at different speeds in
crystals.

To solve the equations of motion, we must place another constraint on the
system. A common constraint is periodic boundary conditions. When solving for
the energy of the system, the result is
 \begin{equation*}
	 E = \sum_{j}{\frac{\hbar \omega_j}{e^{\beta \hbar \omega_{j}} -1}} = 
	 \frac{Na}{\pi} \int_{0}^{\pi / a}{\frac{\hbar \omega(k)}{e^{\beta \hbar
	 \omega(k)} -1} \d{k}}.
\end{equation*}
The bounds on the integral are from the fact that $\omega(k) = \omega(|k|)$.
Using chain rule we can eliminate $k$,
\begin{align*}
	\d{k} &= \frac{\d{k}}{\d{\omega}} \d{\omega} = \frac{\d}{\d{\omega}}{\left[
		\frac{2}{a}\sin^{-1}{\left(\frac{\omega}{\omega_{max}}\right)}\right]}
		\d{\omega}\\
		&= \frac{2 \d{\omega}}{a(\omega^2_{max} - \omega^2 )^{1 /2}}.
\end{align*}
This can be used to rewrite the above integral in terms of $\omega$ with limits
$0, \omega_{max}$.

\subsection{Diatomic 1D Crystal}
The Hamiltonian for a diatomic alternating crystal is
\begin{equation*}
	\H = \sum_{j=1}^{N}{\frac{m_1}{2}\zeta^2_{2j} + \frac{m_2}{2}\zeta^2_{2j-1}}
	+ \frac{f}{2} \sum_{j=1}^{N}{{[ {(\zeta_{2j} - \zeta_{2j-1})}^2 +
	{(\zeta_{2j+1} - \zeta_{2j})}^2 ]}}.
\end{equation*}
This system leads to one set of equations of motion for each particle type. The
solution is
\begin{align*}
	\omega^2 &= \omega^2_0 {\left[ 1 \pm {\left(1 - \frac{4m_1 m_2
	\sin^2{\phi}}{{(m_1 m_2 )}^2}\right)}^{1 / 2}\right]}\\
	\omega^2_0 = \frac{f}{\mu}.
\end{align*}
One interesting part of the solution, is that two sets of frequencies exist
depending on if $+$ or $-$ is chosen. When atoms move together, the branch is
called the acoustic branch. When neighboring atoms and displaced in opposite
directions, it is called the optical branch. If the atoms have differing
charges, the optical branch can absorb and emit light. This is usually in the
infrared spectrum.

\section{Defects}%
\label{sec:crystal_defects}
We shall now look at the energetic and entropic effects on defects in solids.
Defects in a solid phase are energetically costly, but as we will see, when
entropic effects are taken into consideration, they are almost certain.

\subsection{Lattice Vacancy}
A missing atom in a lattice site is called an \textit{Schottky defect}. Assume
that a \textit{Schottky defect} cost a crystal an energy of $\varepsilon_s$.
Then the free energy as a function of defects becomes,
\begin{align*}
	A(n) &= E - TS\\
		 &= n \varepsilon_s - kT \ln{\frac{N!}{n!(N-n)!}},
\end{align*}
where the argument in the logarithm is the number of unique ways to distribute n
\textit{Schottky defect}. If we take the derivative with respect to $n$ we get,
\begin{align*}
	\frac{\partial A}{\partial n} &= \varepsilon_s - kT\frac{\partial}{\partial
	n}{\left[ \ln{N!} - \ln{n!} - \ln{(N - n)!}\right]}\\
	&\approx \varepsilon_s - kT\frac{\partial}{\partial n}{\left[ -\ln{n} + n
	-\ln{(N-n)!} + N - n \right]}\\
	&\approx \varepsilon_s + kT{\left(\frac{1}{n} - \frac{1}{N-n}\right)}.
\end{align*}

\subsection{Interstitial Defect}
When an atom is not placed correctly on a lattice site it is called an
\textit{Frenkel defect}. If $N'$ interstitial sites exist then $A(n)$ is
\begin{equation*}
	A(n) = n\varepsilon_i - kT \ln{\left[\frac{N!}{n!(N-n)!} \cdot
	\frac{N'!}{n!(N' - n)!}\right]}.
\end{equation*}
The minimization of free energy can similarly be done in this case.

\subsection{Summary}
The dominate type of defect can be determined using a careful measure of
density. Frenkel defects do not change density while Schottky defect since they
move particles to the surface do. The diffusion of atoms through a lattice
depends in large measure to the concentration of defect. If moving to a vacant
site takes an energy $\varepsilon$ then the probability an atom has an energy to
overcome the barrier is $e^{-\varepsilon /kT}$. An atom with a fequency of $\nu$
then has a time probability of jumping $p \approx \nu e^{-\varepsilon /kT}$. The
flux of atoms across one lattice constant is then $pa (\d{c}/\d{x})$ where c is
the number of impurities. If  $n$ represents the linear concentration of
impurities then $c = an$ and $j = -pa^2 (\partial n / \partial x)$. Comparing
this to Fick's law,
\begin{equation*}
	j = -D \frac{\partial n}{\partial x},
\end{equation*}
we see that $D = pa^2 \approx \nu a^2 e^{-\varepsilon /kT}$.

Colors in alkali halide crystals is completely a result of defects. The defects
are appropriately known as \textit{color centers}. One type F centers occur when
an alkali metal vapor is introduced to the alkali halide crystal. The alkali
metal get incorporated and separated from its electron. The electron becomes
effectively attached to its lattice site and can absorb visible light.
