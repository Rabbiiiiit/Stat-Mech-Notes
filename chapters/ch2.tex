% Chapter 2 Notes

\subsection{Assumptions}

\begin{itemize}

	\item No uncertainty in energy states.

	\item Each system is a pure system.

	\item An isolated system is equally likely to be in any available state.

	\item The average of a property over all available microstates is the
		macroscopic value.

\end{itemize}

\section{MicroCanonical}\label{sec:ch2MicroCanonical}

For a system with constant number, volume, and energy, the number of microstates
is \[ \Omega = \Omega(N, V, E).\]

Now take an ensemble or collection of systems like this with the number of
systems in the ensemble being $A$. Then, \[ E_{tot} = A E
\]\[N_{tot} = A N \] \[ V_{tot} = A V.\]  Since we have no reason to favor one
system over the other, we assume each state with the same $N, V, E$ is equally
probable. This is known as the principle of equal \textit{a priori}
probabilities. 

\section{Canonical}\label{sec:Canonical}

\subsection{Deriving the Partition Function}

Now lets take a ensemble of systems with the same temperature, volume, and
number. The ensemble is put into a heat bath until it is fully equilibrated,
and then moved into perfect thermal insulation. The entire ensemble then becomes
a system in the microcanonical ensemble.

We know that the system must obey,
\[ A = \sum_j{a_j} \]\[ E_{tot} = \sum_j{a_j E_j},\]
where  $a_j$ is the occupancy number of state j. We generally do not know the
occupancy number of states or number of states, however.

We do know (assumption) that for the microcanonical ensemble each state is
equally likely. This means that each state for the entire canonical ensemble is
equally likely since the entire canonical ensemble is a system in the
microcanonical ensemble; that is all combinations of occupancy numbers is
equally likely.

The number of states with distribution $\vec{a}$ of occupancy numbers is given
by
\[ W(\vec{a}) = \frac{A!}{ \prod_{j}{a_j!}}.\]
This a multinomial coefficient meaning that at large $A$ it is very sharply
peaked; in fact, arbitrarily so at large enough $A$. The probability of being in
state j is,
\[ P_j =  \frac{\sum_a{W(\vec{a})a_j(\vec{a})}}{A \sum_a{W(\vec{a})}}
= \frac{\bar{a_j}}{A}.\]
At infinite number of systems in an ensemble, all contributions besides the most
common become completely negligible, thus
\[ P_j = \frac{\bar{a_j}}{A} = \frac{a_j^{*}}{A}.\] 

We now have a maximization problem (maximizing the weight function $W$) with the
following constraints.
\[ A = \sum_{j}{a_j} \]
\[ E_{tot} = \sum_{j}{a_j E_j}.\]

Since $\ln$ is a monotonic function and, thus, will preserve maxima and minima of
inputted functions, $\ln(W(\vec{a}))$ can be maximized instead of $W(\vec{a})$.
For further calculations it behooves us to recognize that
\[ \ln(W) = \ln(A!) - \sum_j{\ln(a_j!)}.\]
Note the first term is constant and the second is approximately
\[-\sum_j{a_j \ln(a_j)} + \sum_{j}{a_j} = -\sum_j{a_j \ln(a_j)} + A.\]
Using this we take the partial differential,
\[ \frac{\partial ( \ln W(\vec{a}) - \alpha \sum_{j}{a_j} - \beta \sum_j{a_j
E_j})}{\partial a_{k}} = 0, \]
for each $k$ and find,
\[ \ln(a^*_k)+ \alpha + 1 + \beta E_k = 0 \]
or
\[ a_k^* = e^{-\alpha^{'}}e^{-\beta E_{k}}, \text{where} \alpha^{'} = \alpha +
1. \]
Summing over all $(a_j^*)$ gives,
\[ A = e^{ -\alpha^{'}}\sum_{j}{e^{ - \beta E_j}}. \]
Finally,
\[ P_j = \frac{e^{-\alpha^{'}}e^{-\beta E_{k}}}{e^{ -\alpha^{'}}\sum_j{e^{
-\beta E_{j}}}} = \frac{e^{-\beta E_k}}{\sum_j{e^{ -\beta E_j}}}.\]

We call the denominator $Q$ and label it the partition function. We now show
the ensemble energy and some of its derivatives.

\subsection{Finding $\beta$}

\[\bar{E} = \frac{\sum_{j}{E_j e^{-\beta E_{j}}}}{Q}.\]
\[ p_j = - \left(\frac{\partial E}{\partial V}\right)_N.\]
\[ \bar{p} = - \frac{\sum_j{\left( \frac{\partial E}{\partial V} \right)_{N}
e^{-\beta E_j}}}{Q}.\]

We can then find,
\[ \left( \frac{ \partial \bar{E} }{ \partial V}\right)_{N, \beta} = -\bar{p}
+ \beta \overline{Ep} - \beta \bar{E} \bar{p} \] and
\[ \left( \frac{ \partial \bar{p} }{ \partial \beta }\right)_{N, V} =
\bar{E}\bar{p} - \overline{Ep}.\]

Comparing the thermodynamic and ensemble averages we see,
\[\left( \frac{ \partial \bar{E} }{ \partial V}\right)_{N, \beta} +
\beta \left( \frac{ \partial \bar{p} }{ \partial \beta }\right)_{N, V} =
- \bar{p},\]

\[\left( \frac{ \partial E }{ \partial V}\right)_{N, T} -
T \left( \frac{ \partial p }{ \partial T }\right)_{N, V} = - p.\]

One can see from this that $ \beta = \frac{k}{T} $. k will be the same for any
system as can be seen from a thought experiment where the ensemble consists of
two system composites with the same $N, V, T$. They individual systems will be
in equilibrium and from prior calculations the partition function and
probability functions are completely separable. This requires they have the same
constant k.

\subsection{Entropy}

Let \[f = \ln Q.\] Then, \[ df = \left( \frac{ \partial f }{ \partial \beta }
\right)_{E_{j}} d \beta + \sum_{k}{\left( \frac{ \partial f }{ \partial E_{k}}
\right)_{\beta, E_{j \neq k}} d E_{k}}.\] 
This derivative is equivalent of slightly changing the volumes of all systems
which changes the energy levels (which is nothing but PV work) and slightly
change the temperature before isolated the ensemble again.

This becomes with some manipulation,
\[ df = -\bar{E} d \beta - \beta \sum_{k}{P_{k} dE_{k}} \]
\[ d(f + \beta \bar{E}) = \beta \left( d \bar{E} - \sum_{k}{P_{k} dE_{k}}
\right).\]

The first term is the change in average system energy, while the second is the
average change of energies in corresponding energy levels. The first is a total
energy change while the second is average reversible work. Thus, the quantity is
the average reversible heat that of system. This requires that
$\beta$ is an integrating factor of $f$.  \[ d(f + \beta \bar{E}) = \beta
q_{rev} \]
\[ d\left(f + \frac{ \bar{E} }{kT} \right) = \frac{q_{rev}}{kT}.\]
Thermodynamically the left side must be $ dS/k $, so finally,
\[ dS = \frac{d \bar{E}}{T} + k d( \ln Q).\]

\subsection{Spontaneous Processes}
One can think of spontaneous processes as processes that occur because an
increase in available states.

We take the known relation,
\[ A(N, V, T) = -k T \ln Q(N, V, T).\]
In addition, one can also recognize that a summation over energy levels rather
than states can be done.
\[ Q = \sum_{l}{\Omega(N, V, E_{l}) e^{- \beta E_{l}}}.\]
For a spontaneous process we require that or that available states can only be
increased,
\[ \Omega_2 \geq \Omega_1.\]
Then,
\[ Q_2 - Q_1 = \sum_{l}{(\Omega_2 - \Omega_1) e^{-\beta E_{l}}} \geq 0.\]
Finally,
\[ \Delta A = A_2 - A_1 = - kT \ln \frac{Q_2}{Q_1} < 0.\]
